The process of configuring a model on the \graicore{} is the reading of the relevent data from the external memory, transferring it to the \graicore{}'s neuron cores and writing the data to the SRAMs.

A compiled model, that is to be written to the \graicore{}, is structured as separate sections of data, with each section containing a certain type of data.
These sections of data are as follows:
\begin{description}
    \item[Weights] 
    The numerical values assigned to the connections between neurons in the neural network
    \item[Masks] 
    A mask is a single bit that is associated to one weight.
    Masks are necessary to identify whether a weight is pruned or not.
    Weights that are pruned (i.e., has an exact value of $0$) are not configured on the \graicore{} at all, saving memory and energy (since it isn't written to or read from the SRAM when processing)
    \item[States] 
    Acts as a buffer space for the accumulation of activation (output) data of layers, which also initially includes the bias values.
    \item[Headers] 
    Contains information about the operation types of the layers, various information about the parameters, and more.
    \item[Axons] 
    Axons contain information about the destinations for the new event(s) injected into the network
    \item[Queues] 
    Acts as an input buffer for storing incoming events from other neuron cores 
\end{description}

From \cref{ch:6}, we observed that, in the configuration process, reading from the external LPPDR5X memory consumes most of the energy.
In this chapter, we look at a few potential means to improve the configuration energy.
