\section{Conclusion}
As AI models continue to grow in size and complexity, existing AI accelerators, such as the \graicore{}, must adapt to handle these demanding workloads, calling for novel approaches to enhance their capabilities.
This research explores effective approaches to enable the execution of multiple and large neural networks on \graicore{} using external memory.
Beginning with a comprehensive evaluation of the \graicore{}'s \confignoc{}, its limitations in handling rapid (re)configuration for such models were identified.
A subsequent analysis of the requirements for the execution of these models laid the foundation for the solutions developed in this work.
By strategically widening data links, doubling the injection points, and optimizing the data packet format, this research achieved a remarkable 16x increase in effective write bandwidth for the \confignoc{}, enabling the \graicore{} to fully leverage the potential of high-speed external memory.
This breakthrough not only unleashes the potential for running larger models on the memory-constrained \graicore{} but also enables rapid (re)configuration of the device to switch between different models at runtime.
Recognizing the challenges of efficiently partitioning large models for external memory execution, this work provides a formal problem definition and demonstrates a practical partitioning strategy using a heuristic approach, exemplified by the partitioning of the ResNet-101 model.
The analysis of energy consumption provides valuable insights into the energy dynamics of two external memory technologies when reconfiguring the \graicore{} with various example models
Notably, LPDDR5X memory, the configuration process accounts for 23\% of the total energy consumed (including both configuration and processing), while this figure drops to 12\% with PCM.
To further enhance energy efficiency, this thesis introduces effective methods to minimize data transfer during reconfiguration which includes the avoidance of transfers of redudant data.
This thesis lays the groundwork for the next generation of \graicore{}, empowering it to tackle increasingly complex AI workloads with enhanced efficiency and flexibility.
