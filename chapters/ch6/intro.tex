To obtain an insight into the energy consumption with the introduction of the \confignoc{} changes in \cref{ch:5} and model configuration using external memory we provide a comprehensive analysis.

The investigation presented specifically targets dynamic energy consumption.
This focus allows for a detailed analysis of the energy efficiency of fundamental memory operations, including read/write accesses and data transfers.
By characterizing the dynamic energy profiles, this study aims to identify key areas for potential optimization and energy reduction.

It is important to note that this study is limited to dynamic energy consumption and does not consider static energy.
Static energy, the energy consumed when the system is in an idle state, involves a separate set of complex factors, such as leakage currents and power gating techniques \cite{haj-yahyaStaticPowerModeling2018}.
Due to the intricacies associated with static energy analysis, it falls outside the defined scope of this study, which prioritizes a thorough examination of dynamic energy behaviors.
However, the potential contribution of static energy to the overall energy budget is acknowledged.

% A comprehensive understanding of energy efficiency necessitates investigating static energy consumption, which remains an important area for future research.

When configuring the \graicore{} with a model and processing an input frame with a model, the major components that consume energy are as follows:
\begin{itemize}
    \item Configuration
    \begin{itemize}
        \item Reading data from the external memory
        \item Transferring data from the external memory to the \confignoc{}
        \item Phits traversing (i.e., hops) through the \confignoc{}
        \item Writing data to the neuron core's SRAM
    \end{itemize}
    \item Processing
    \begin{itemize}
        \item Processing the input frame by the model
    \end{itemize}
\end{itemize}
