\section{Conclusion}
This chapter conducted an analysis of the energy consumption associated with model reconfiguration on the \graicore{} using off-chip memory.
By focusing on dynamic energy consumption, which is the energy consumed during active operation, this research aims to identify key areas for potential optimization and energy reduction.

The results show that, with using the DDR interface, LPDDR5X memory exhibits a high proportion of energy dedicated to read operations (65\%), whereas PCM exhibits a much higher proportion of energy dedicated to the DDR PHY (43\%).
Additionally, the AXI bus is a significant energy consumer in both LPDDR5X and PCM.
The study highlights the potential benefits of integrating a DDR interface as an external memory solution for the \graicore{}, as it can bring about substantial performance improvements.

When including the processing energy of a frame, it was found that the configuration energy is an important factor in the overall energy consumption.
The analysis revealed that PCM exhibits lower proportions of configuration and processing energy compared to LPDDR5X, suggesting its potential for reduced energy consumption.
However, it is essential to note that this study focuses solely on dynamic energy, and static energy remains an important area for future research.
