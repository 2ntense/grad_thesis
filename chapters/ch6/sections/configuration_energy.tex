\section{Configuration energy}
Since we have access to various energy parameters for DDR memory, we will use DDR memory as an example.

\begin{figure}[hbtp]
    \centering
    \includegraphics[width=0.8\linewidth]{assets/ddr_graicore_block_diagram.pdf}
    \caption{
        Interconnection of the DDR device and \graicore{}
    }
    \label{fig:ddr_graicore_block_diagram}
\end{figure}

A typical system with DDR as external memory looks as shown in \cref{fig:ddr_graicore_block_diagram}.
% TODO add image
\begin{itemize}
    \item The \textit{DDR device} where the (model) data is read from.
    \item The \textit{DDR PHY} that connects the \textit{DDR device} and \textit{DDR controller}.
    \item The \textit{DDR controller} that handles the read/write requests from the \textit{\graicore{}}.
    \item The \textit{AXI bus} that connects the \textit{DDR controller} and \textit{\graicore{}}.
    \item The \textit{\graicore{}} that transfers and writes data the SRAMs.
\end{itemize}

% It has been established that performing a hop on the \eventnoc{} consumes around \SI{129}{fJ}.
% These hops are for phits of 32 bits.
% For the \confignoc{} (which uses phits of 16 bits), we are interested in the energy use for transferring phits of 16 bits. We can approximate this value by dividing the value for 32 phit hops into two. I.e., 129 fJ / 2 = 64.5 fJ. 

% Furthermore, it has been established that for the SRAM to write 64 bits, it consumes \SI{2.490}{pJ}.
% The SRAM system has the property to write 64 bits at once by first buffering four consecutive phits.
% This improves efficiency. 

For the analysis, we assume that we are using the newly proposed \confignoc{} in \cref{sec:proposed_noc}. 
With the new packet format, a packet contains at most 65 data phits of 64 bits each.
That is, a maximum of 520 bytes\footnote{$65 \times \frac{\SI{64}{b}}{8} = \SI{520}{B}$} of payload per packet.
Furthermore, an additional injection point is introduced (see \cref{fig:segmentation_example_2}).
Next to the injection point connected to the router on the bottom left of the \confignoc{}, a new one is added and connected to the router six positions up.

To estimate the energy cost for configuration, we require the following information from a model:
\begin{itemize}
    % \item Amount of data to read from the external memory
    \item Amount of data to transfer to the SRAMs
    \item Core destination of the data
\end{itemize}

The amount of data to transfer determines how much data will be read, transferred and written.
We are assuming that the same amount of data read from the external memory is written to the SRAMs.
The location (specific neuron core) of the data determines the amount of hops the data has to perform in the \confignoc{}.
The amount of data to write can also be used to determine the configuration time.
The configuration time is calculated by dividing the amount of data with the write bandwidth.
The configuration time is used for computing the energy for the \textit{DDR PHY}, \textit{DDR controller} and \textit{AXI bus}.

The amount of data to be written and to which neuron cores depends on how the compiler has performed the mapping on the \graicore{}. % explain
Phits that needs to be transferred to neuron cores further away from an injection point will require more hops to reach, and therefore consume more energy than neuron cores closer to an injection point.
The amount of data that needs to be written to each core differs.
We can retrieve this information from the compiler.

For estimating the energy for the \graicore{} component, we consider the \confignoc{} and the SRAMs.
The \confignoc{} consumes energy by transferring the phits to its destinations via one or multiple hops through the NoC.
The SRAM consumes energy by writing the data to its banks.
A single hop through the \confignoc{} with a 64 bit phit consumes \SI{0.258}{pJ}.
Writing back 64 bits to an SRAM consumes \SI{4.980}{pJ}.

\begin{table}[hbtp]
\centering
\begin{tabular}{@{}lll@{}}
\toprule
\textbf{Component}      & \textbf{Usage}  &  \\
\midrule
DDR device              & \SI{0.11}{nJ/B} &  \\
DDR PHY                 & \SI{133}{mW}    &  \\
DDR controller          & \SI{20}{mW}     &  \\
AXI bus                 & \SI{80}{mW}     &  \\
NoC hop (\SI{64}{b})    & \SI{0.258}{pJ}  &  \\
SRAM write (\SI{64}{b}) & \SI{4.980}{pJ}  &  \\
\bottomrule
\end{tabular}
\caption{Energy parameters for each of the major energy consuming components. The shown power numbers for \textit{DDR PHY}, \textit{DDR controller} and \textit{AXI bus} are when at full capacity (i.e., best-case scenario).}
\label{tab:energy_parameters_ddr}
\end{table}

\subsection{Calculation}
The configuration energy consists of the reading, transferring and writing of data from the external memory to the \graicore{}'s SRAMs. 

Let $C$ be the set of tuples holding the coordinates of every core:
\begin{equation*}
    C = \{\,\left(x,y\right) \in \mathbb{N}^2 \mid 1 \leq x \leq 12 \wedge 0 \leq y \leq 11 \,\} 
\end{equation*}

Notice that the x-coordinate and y-coordinate starts at index $1$ and $0$ respectively.

\Cref{fig:model_data_heapmap} shows for a $80\%$ pruned version of ResNet-50\footnote{Internally named \texttt{resnet50\_pruned80\_star}} the amount of data that needs to be written to each of the 144 neuron cores.
We observe that the data is not uniformly distributed across the SRAMs.
Therefore, we require information how much data needs to be transferred to each SRAM.

\begin{figure}[hbtp]
    \centering
    \includegraphics[width=0.8\linewidth]{assets/model_data_heatmap.png}
    \caption{Amount of data to be written to each core for the ResNet-50 model (80\% pruned).}
    \label{fig:model_data_heapmap}
\end{figure}

Let $D$ be a matrix of $12 \times 13$ with $D_{i,j}$ denoting the amount of bytes to be written to core $\left( i,j \right)$.
$D$ has an additional column due to the x-coordinates not starting from index $0$.
Its left-most column is unused (i.e., $D_{0,0}, D_{0,1}, \cdots, D_{0,11}$).
This matrix can be constructed from the artifacts outputted by the compiler.

The number of phits to be transferred through the \confignoc{} influences the total energy costs.
In particular, the amount of phits to be transferred affects the number of hops to be taken in total and the amount of data to be written to the SRAMs.
Therefore, we need to determine how many phits needs to be transferred to each neuron core.
% The number of phits to be transferred influences the energy cost in the \confignoc{} hops and SRAM writes.
% A packet can contain up to 520 bytes\footnote{$65 \times \frac{64}{8}$} of payload data. 
A packet can contain up to 65 data phits, that is $65 \times \SI{64}{b} = \SI{520}{B}$ of payload data.
% If we transfe
Suppose we need to transfer $d$ bytes to a neuron core, we then require a total of $\left\lfloor \frac{d}{520} \right\rfloor$ packets with 65 data phits.
% If $d \bmod 520 > 0$, then there is an additional packet consisting of $\left\lceil \frac{\left( d \bmod 520 \right) \times 8}{64}\right\rceil$ data phits.
If $\left( d \bmod 520 \right) > 0$, then there is an additional packet for the remaining $\left( d \bmod 520 \right)$ bytes of data.
The remaining packet will consist of $\left\lceil \frac{d \bmod 520}{8}\right\rceil$ data phits.
Note that each packet also contains a single phit of 64 bits for the header information.

The total energy cost for configuring the \graicore{} can be estimated with the following equation:
\begin{equation}
    E_{\textrm{config}} = E_{\textrm{ext\_mem}} + E_{\textrm{noc}} + E_{\textrm{write}}
\end{equation}

With:
\begin{align*} 
E_{\textrm{ext\_mem}} &= 
        \sum_{c \in C}^{}{E_\textrm{read\_ext\_mem}(D_c) + E_{\textrm{send\_to\_noc}}(D_c)} \\
E_{\textrm{noc}} &=
    E_{\textrm{hop}} \times \sum_{c \in C}^{}{N_\textrm{hops}(c) \times p_{\textrm{total}}(D_c)} \\
E_{\textrm{write}} &=
    E_{\textrm{sram\_write\_64b}} \times \sum_{c \in C}^{}{p_{\textrm{data}}(D_c)}
\end{align*}

And:
\begin{align*} 
N_{\textrm{hops}}(x,y) &=
    \begin{cases} 
        x + y & \textrm{if } 0 \leq y \leq 5 \\
        x + y - 6 & \text{if } 6 \leq y \leq 11
    \end{cases}
\\
p_{\textrm{total}}(d) &=
    \left\lfloor \frac{d}{520} \right\rfloor \times (65 + 1) + \left\lceil \frac{d \bmod 520}{8} \right\rceil + 1 =
    \left\lceil \frac{d}{8} \right\rceil + \left\lfloor \frac{d}{520} \right\rfloor + 1 
\\
p_{\textrm{data}}(d) &=
    \left\lfloor \frac{d}{520} \right\rfloor \times 65 + \left\lceil \frac{d \bmod 520}{8} \right\rceil =
    \left\lceil \frac{d}{8} \right\rceil
\end{align*}

\begin{eqexpl}[15mm]
    \item{$p_{\textrm{total}}(D_c)$} total phits (includes headers) for transferring $D_c$ bytes
    \item{$p_{\textrm{data}}(D_c)$} total data phits (excludes headers) for transferring $D_c$ bytes
    \item{$N_{\textrm{hops}}(c)$} the amount of hops required to reach a neuron core at coordinate $c$, starting from the router closest to the injection point. It has two sub-functions due to the new \confignoc{}'s dual injector architecture
    \item{$E_{\textrm{read\_ext\_mem}}(D_c)$} energy for reading $D_c$ bytes from the external memory
    \item{$E_{\textrm{send\_to\_noc}}(D_c)$} energy for sending $D_c$ bytes from the external memory to the \confignoc{}
    \item{$E_{\textrm{hop}}$} energy for performing a single hop in the \confignoc{}
    \item{$E_{\textrm{sram\_write\_64b}}$} energy for writing back 64 bits to a neuron core's SRAM
\end{eqexpl}

