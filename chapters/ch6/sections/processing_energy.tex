\section{Processing energy}
The processing energy is the energy consumed when the GrAICore is processing input frames.
% Note that the GrAICore does not make use of the config NoC to process input frames.
Note that the GrAICore uses the event NoC for communication between nodes when processing input frames, the config NoC is not involved in this process.

The processing energy can be estimated with the following equation:
\begin{equation}
    E_{\textrm{frame}} = \textrm{avg\_util} \times \textrm{cores\_used} \times \textrm{processing\_latency} \times \SI{21}{mW}
\end{equation}

These parameters can be obtained by simulating the model with \textit{GrAIPEFRUIT}, an in-house simulator for estimating inference performance.
Processing latency (\textrm{processing\_latency}) is the time the model takes to fully process a single frame, from input to output.
Cores used (\textrm{cores\_used}) is the amount of neuron cores that were used to execute the model.
Average utilization (\textrm{avg\_util}) is the average percentage of the time the cores were active while the frame was processed. 
The constant \SI{21}{mW} is the approximated power usage of a single neuron core while it's being fully utilized.
This constant is obtained from internal RTL simulations.

Since the config NoC is not involved in the processing of an input frame, any change to the config NoC does not influence the processing energy.
A significant factor (other than hardware changes) that may affect the processing energy is the mapping performed by the compiler on the original model. 
Different mappings on the same model have an effect on the average utilization and the number of neuron cores used, which in turn influences the processing latency.

