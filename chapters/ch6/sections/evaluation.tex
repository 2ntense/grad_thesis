\section{Evaluation}
To develop a preliminary understanding of the power contributions of the model reconfiguration from external memory, we evaluate a variety of practical models.
The power is the configuration of a model and a single inference with the model at \SI{60}{FPS}.
Effectively, this means that the same model is configured on the GrAICore 60 times per second and also used to proces an input frame 30 times per second.

We will be looking at two different external memory technologies, LPDDR5X and PCM.
We assume we use the DDR protocol for transferring data to the AXI bus that is connected to the GrAICore.
For the DDR protocol to work, we require an DDR PHY and DDR controller.
These components both add to the energy when transferring data to the GrAICore.

\begin{table}[hbtp]
    \centering
    \begin{tabular}{@{}lrl@{}}
    \toprule
    \textbf{Component} & \textbf{Value} & \textbf{Unit} \\
    \midrule
    LPDDR5X read             & 36.00 & pJ/B \\
    PCM read                 & 5.29 & pJ/B  \\
    DDR PHY                  & 10.39 & pJ/B \\
    DDR Controller           & 1.56 & pJ/B  \\
    AXI Bus                  & 6.25 & pJ/B  \\
    Config NoC hop (64 bits) & 0.26 & pJ    \\
    SRAM write (64 bits)     & 4.98 & pJ    \\ \bottomrule
    \end{tabular}
    \caption{Energy parameters of PCM and LPDDR5X memory, DDR interface and GrAICore that are relevant for model configuration}
    \label{tab:energy_parameters}
\end{table}

We look at the models as shown in \cref{tab:example_models_stats}.
It also shows the processing latency, the number of cores used and the average utilization of the cores of each (mapped) model.
These statistics are used for computing the processing energy.
The ``to write'' columns show the total amount of bytes that is to be transferred to the GrAICore.
Note that to get a more accurate energy number for configuration, we need the amount of bytes to be transferred to each core individually.
This information is available, and will be used for the evaluation.

\begin{table}[]
\centering
\begin{tabular}{@{}lrrrrr@{}}
\toprule
\textbf{Model}          & \textbf{\begin{tabular}[c]{@{}l@{}}Latency\\ (ms)\end{tabular}} & \textbf{Cores} & \textbf{Avg util.} & \textbf{\begin{tabular}[c]{@{}l@{}}To write\\ (MiB)\end{tabular}} \\ \midrule
efficientnet            & 1.642                                                           & 144            & 50.89\%            & 16.15                                                             \\
mobnetv2                & 1.296                                                           & 144            & 40.78\%            & 10.67                                                             \\
hand\_tracker           & 1.480                                                           & 144            & 36.62\%            & 12.33                                                             \\
hand\_detector          & 4.809                                                           & 144            & 68.03\%            & 19.12                                                             \\
resnet50                & 5.765                                                           & 144            & 36.35\%            & 29.58                                                             \\
resnet101\_p0           & 7.080                                                           & 144            & 33.40\%            & 28.75                                                             \\
resnet101\_p1           & 2.647                                                           & 143            & 36.42\%            & 20.06                                                             \\
resnet101\_p2           & 4.011                                                           & 144            & 33.12\%            & 30.44                                                             \\
resnet101\_p3           & 2.343                                                           & 143            & 33.55\%            & 17.71                                                             \\
resnet101\_p4           & 2.040                                                           & 143            & 26.19\%            & 15.54                                                             \\
resnet101\_pruned\_p0   & 6.552                                                           & 144            & 38.10\%            & 15.87                                                             \\
resnet101\_pruned\_p1   & 3.245                                                           & 143            & 34.31\%            & 10.43                                                             \\
resnet101\_pruned\_p2   & 4.501                                                           & 143            & 33.83\%            & 16.00                                                             \\
resnet101\_pruned\_p3   & 2.114                                                           & 143            & 38.08\%            & 10.22                                                             \\
resnet101\_pruned\_p4   & 2.437                                                           & 143            & 24.25\%            & 7.69                                                              \\
\bottomrule
\end{tabular}
\caption{Model statistics}
\label{tab:example_models_stats}
\end{table}

As an example, we demonstrate a calculation for the ResNet-50 model.
The processing energy is calculated as follows:
\begin{equation}
    E_\textrm{proc} = 0.3635 \times 144 \times \SI{5.765}{ms} \times \SI{21}{mW} = \SI{6.34}{mJ}
\end{equation}

\Cref{fig:model_data_heapmap} shows for the mapped ResNet-50 the amount of data to be sent to each individual core.
We use this information to construct matrix $D$ and the configuration energy $E_\textrm{conf}$.
We obtain the values as shown in \cref{tab:resnet50_energy}.
Configuration with LPDDR5X of the ResNet-50 model on the GrAICore consumes \SI{1.71}{mJ} and with PCM \SI{0.76}{mJ}.

\begin{table}[hbtp]
    \centering
    \begin{tabular}{@{}ll@{}}
    \toprule
    \textbf{Component} & \textbf{Value} \\
    \midrule
    $\ereadlpddr5x$ & 1116.8 \\
    $\ereadpcm$ & 164.2 \\
    $\ephy$ & 322.4 \\
    $\ectrl$ & 48.5 \\
    $\eaxi$ & 193.9 \\
    $\enoc$ & 9.2 \\
    $\ewrite$ & 19.3 \\
    \bottomrule
    \end{tabular}
    \caption{All values are in \SI{}{\micro\joule}}
    \label{tab:resnet50_energy}
\end{table}

% Then, performing a single configuration and a single inference, the system will consume \SI{8.05}{mJ} for LPDDR5X and \SI{7.09}{mJ} for PCM in total.
Then, the total energy consumption when performing a single configuration and a single inference is \SI{8.05}{mJ} for LPDDR5X and \SI{7.09}{mJ} for PCM.

\begin{figure}[hbtp]
    \centering
    \subcaptionbox{LPDDR5X\label{fig:pie_resnet50_conf_lpddr5x}}{
        \import{assets/power_analysis}{pre}
\begin{tikzpicture}
    \pie[
        radius=1.8,
        text=pin,
        color = {blue!60, blue!50, blue!40, blue!30, blue!20, blue!10},
        before number=\printonlylargeenough{10},
        after number=\ifprintnumber\%\fi
    ]{
        % 64.9/$\eread$,
        % 19.3/$\ephy$,
        % 2.8/$\ectrl$,
        % 11.3/$\eaxi$,
        % 0.5/$\enoc$,
        % 1.1/$\ewrite$
        64.9/$\eread$,
        19.3/$\ephy$,
        % 2.8/$\ectrl$,
        11.3/$\eaxi$,
        % 0.5/$\enoc$,
        % 1.1/$\ewrite$
        4.4/$\textrm{other}$
    }
\end{tikzpicture}

    }
    \hfill
    \subcaptionbox{PCM\label{fig:pie_resnet50_conf_pcm}}{
        \import{assets/power_analysis}{pre}
\begin{tikzpicture}
    \pie[
        radius=1.8,
        text=pin,
        color = {blue!60, blue!50, blue!40, blue!30, blue!20, blue!10},
        before number=\printonlylargeenough{10},
        after number=\ifprintnumber\%\fi
    ]{
        % 21.4/$\eread$,
        % 43.3/$\ephy$,
        % 6.3/$\ectrl$,
        % 25.3/$\eaxi$,
        % 1.2/$\enoc$,
        % 2.5/$\ewrite$
        21.4/$\eread$,
        43.3/$\ephy$,
        % 6.3/$\ectrl$,
        25.3/$\eaxi$,
        10.0/$\textrm{other}$
        % 1.2/$\enoc$,
        % 2.5/$\ewrite$
    }
\end{tikzpicture}

    }
    \caption{ResNet-50 configuration energy distribution}
    \label{fig:resnet50_conf_energy_distribution}
\end{figure}

% \Cref{TODO} shows that for LPDDR5X memory, reading from the memory consumes most of the energy in the configuration process.
% While the DDR PHY consumes the most energy when PCM is used.

Looking at the configuration energy, with LPDDR5X, most of the energy is consumed by reading from the exteral memory (see \cref{fig:resnet50_conf_energy_distribution}).
While for PCM, most of the energy is consumed by the DDR PHY.

The energy distribution between configuration and processing.
Shown in \cref{fig:resnet50_conf_proc}, the configuration of the ResNet-50 model occupies around 21\% with LPPDDR5X and 11\% with PCM of the total energy consumption (processing and configuration).
Of the models as listed in \cref{tab:example_models_stats}, on average, around 23\% is used for configuration with LPDDR5X memory and 12\% with PCM as external memory (see \cref{fig:example_models_avg_conf_proc}).
When performing configuration and processing at \SI{60}{FPS}, we get the power values as shown in \cref{tab:example_models_power_consumption}.

Ideally, the power consumption of the configuration of a model should be as minimal as possible.
In \cref{chapter:model_configuration_improvements}, we explore techniques for increasing configuration efficiency.

\begin{table}[]
    \centering
    \begin{threeparttable}
        \begin{tabular}{@{}lrrr@{}}
            \toprule
                                       & \multicolumn{2}{l}{\textbf{Conf. power (mW)}} & \textbf{Proc. power (mW)} \\ \cmidrule(l){2-3} 
            \textbf{Model}             & \textit{LPDDR5X}  & \textit{PCM}    & \\ \midrule
            efficientnet               & 56.0              & 24.8            & 151.6 \\
            mobnetv2                   & 37.0              & 16.4            & 95.9 \\
            hand\_tracker              & 42.8              & 18.9            & 98.3 \\
            hand\_detector             & 66.3              & 29.4            & 593.6 \\
            resnet50                   & 102.6             & 45.4            & 380.2 \\
            resnet101\tnote{1}         & 390.2             & 172.8           & 1081.7 \\
            resnet101\_pruned\tnote{1} & 208.9             & 92.5            & 1179.4 \\ \bottomrule
        \end{tabular}
        \begin{tablenotes}
            \item[1] Combination of all parts
        \end{tablenotes}
    \end{threeparttable}
    \caption{Power consumption of configuration (of the same model) and processing of an input frame at \SI{60}{Hz}}
    \label{tab:example_models_power_consumption}
\end{table}

\begin{figure}[hbtp]
    \centering
    \subcaptionbox{LPDDR5X\label{fig:pie_resnet50_conf_proc_lpddr5x}}{
        \begin{tikzpicture}
    \pie[
        radius=1.8,
        text=pin,
        color={blue!60, red!60},
    ]{
        21.2/$\econf$,
        78.8/$\eproc$
    }
\end{tikzpicture}

    }
    \hfill
    \subcaptionbox{PCM\label{fig:pie_resnet50_conf_proc_pcm}}{
        \begin{tikzpicture}
    \pie[
        radius=2,
        text=pin,
        color={blue!60, red!60},
    ]{
        10.7/$\econf$,
        89.3/$\eproc$
    }
\end{tikzpicture}

    }
    \caption{Configuration vs. processing energy for the ResNet-50 model}
    \label{fig:resnet50_conf_proc}
\end{figure}

\begin{figure}[hbtp]
    \centering
    \subcaptionbox{LPDDR5X\label{fig:sunburst_avg_conf_proc_lpddr5x}}{
        \import{assets/power_analysis}{pre}
\begin{tikzpicture}
    \pie[
        radius=2.3,
        text=pin,
        hide number,
    ]{
        1.0/1.0\%,
        14.9/14.9\%,
        4.3/4.3\%,
        2.8/2.8\%
    }
    \pie[
        radius=2.3,
        hide number,
        color={gray, bluehue2, bluehue4, bluehue6},
        before number=\printonlylargeenough{2},
        after number=\ifprintnumber\%\fi
    ]{
        1.0/,
        14.9/,
        4.3/,
        2.8/
    }
    \pie[
        radius=1.8,
        text=inside,
        color={blue!60, red!60},
    ]{
        22.9/$\econf$,
        77.1/$\eproc$
    }
\end{tikzpicture}

    }
    \hfill
    \subcaptionbox*{}[0em]{
        \includegraphics{assets/legend.pdf}
    }
    \hfill
    \subcaptionbox{PCM\label{fig:sunburst_avg_conf_proc_pcm}}{
        \import{assets/power_analysis}{pre}
\begin{tikzpicture}
    \pie[
        radius=2.5,
        text=pin,
        hide number,
    ]{
        1.2/1.2\%,
        2.6/2.6\%,
        5.0/5.0\%,
        3.0/3.0\%
    }
    \pie[
        radius=2.5,
        hide number,
        color={gray, bluehue2, bluehue4, bluehue6},
        before number=\printonlylargeenough{2},
        after number=\ifprintnumber\%\fi
    ]{
        1.2/,
        2.6/,
        5.0/,
        3.0/
    }
    \pie[
        radius=2,
        text=inside,
        color={blue!60, red!60},
    ]{
        11.8/$\econf$,
        88.2/$\eproc$
    }
\end{tikzpicture}

    }
    \caption{Average energy consumption distribution of the models listed in \cref{tab:example_models_stats}}
    \label{fig:example_models_avg_conf_proc}
\end{figure}
    
