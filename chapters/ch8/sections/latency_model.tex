\section{Latency model}
The inference latency of a single partition consists of its processing time and configuration time.
Both the processing time and configuration time of a partition is affected by the compiler parameters.
\begin{equation}
    \lattotal(p_i, \pi_i) = \latconf(p_i, \pi_i) + \latproc(p_i, \pi_i)
\end{equation}
The total inference latency is the sum of the inference latencies of all partitions combined:
\begin{equation}
    \sum_{i=1}^{K} \lattotal(p_i, \pi_i)
\end{equation}
The configuration latency can be approximated as follows:

\begin{equation}
    \latconf(p_i, \pi_i) = \frac{\writeamount(p_i, \pi_i)}{\writebandwidth}
\end{equation}
\eqexplSetIntro{where:}
\eqexplSetItemWidth{6em}
\begin{eqexpl}
    \item{$\writeamount(p_i, \pi_i)$} total bytes written to the cores involved for partition $p_i$ with compiler parameters $\pi_i$.
    \item{$\writebandwidth$} write bandwidth
\end{eqexpl}

\vspace{1em}

It should be noted that this equation gives us an idea of the possible minimum configuration latency.
In reality, the configuration latency is expected to be higher due to various factors such as delays caused by stalls, memory access latencies and resource contention.
