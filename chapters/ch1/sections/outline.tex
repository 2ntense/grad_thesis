\section{Outline}
This thesis is organized in the following way.
\Cref{ch:2} introduces relevant background information about topics that are relevant for the reader for later chapters.
\Cref{ch:3} provides an analysis of the current \confignoc{}, a crucial component of the system that allows for configuration.
It provides an overview of its architecture and discusses its bandwidth.
This chapter provides us with an understanding of its shortcomings and its rooms for improvements.
\Cref{ch:4} provides an analysis of the execution of multiple AI models and large models on the \graicore{}.
It discusses requirements that are required for running these models.
This chapter provides us with the requirements that are needed for the \confignoc{} to be adapted to.
\Cref{ch:5} presents a list of improvements to the \confignoc{} that improves its bandwidth for the configuration process.
The improvements are accompanied with an analysis that examines their effects on bandwidth.
It concludes with a proposed solution for an improved \confignoc{} that is expected to be sufficient for the execution of multiple AI models and large models on the \graicore{}.
\Cref{ch:6} presents a power analysis of model execution with various models with focus on the \confignoc{}.
It gives us an understanding of the added costs of run-time configuration.
\Cref{ch:7} provides a discussion about various methods of improving the efficiency of the configuration process.
Their potential improvements to the configuration process is presented.
\Cref{ch:8} presents a formal problem description of partitioning large models.
This provides us with an understanding of solving the problem of partitioning large models for energy-efficient inference.
\Cref{ch:9,ch:10} present the concluding remarks of the thesis along with future works that are of interest for further research.
