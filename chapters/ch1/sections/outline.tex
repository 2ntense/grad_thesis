\section{Outline}
This thesis begins by introducing key background concepts in \cref{ch:2}, laying the foundation for subsequent analyses.
\Cref{ch:3} investigates the current \confignoc{}, a crucial component for data communication that allows for system configuration.
We will explore its architecture, analyze its bandwidth, and identify areas for bandwidth improvement.

In \cref{ch:4}, we focus on the execution of multiple and large AI models on the \graicore{}.
This analysis reveals the requirements for running these models, which in turn inform the necessary adaptations for the \confignoc{}.

\Cref{ch:5} proposes several \confignoc{} enhancements aimed at improving bandwidth during configuration.
Each improvement is accompanied by a thorough analysis of its impact.
The chapter concludes with a proposed solution for an enhanced \confignoc{} designed to support the execution of multiple and large AI models.

\Cref{ch:6} provides an energy analysis of model configuration, highlighting the costs associated with configuration. 
We then explore various methods for improving the efficiency of the configuration process in \cref{ch:7}, evaluating their potential benefits.

Finally, \cref{ch:8} offers concluding remarks and outlines future works that are of potential interest for future research.




