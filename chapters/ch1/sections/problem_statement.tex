\section{Problem statement}
The problem to be addressed in this thesis is the inability to execute large models and multiple models on the \graicore{}.
The \graicore{} is an edge AI accelerator that was not designed for supporting large models or switching between multiple models, as one of its main design choices---keeping its memory self-contained---limits its total on-chip memory capacity and only supports slow external communication to load data.
Due to this design, the \graicore{} is only configured (or preloaded) once with a model during its boot time.
Furthermore, this design led to an interface between the external host and the \graicore{} in which the bandwidth was not an important factor.

Adding the feature to the \graicore{} to allow it for rapid configuration during run-time can significantly enhance its capabilities.
Firstly, it will allow multiple AI models to be switched between while the \graicore{} is running.
This can be accomplished by rapidly reconfiguring the \graicore{} with a different model while it is in operation.
Deploying multiple AI models can be highly beneficial for creating a multifaceted system capable of handling diverse tasks.
Each model can be specialized for a specific function, allowing the system to be more versatile and responsive to different visual processing requirements.
Secondly, using a technique such as partitioning the neural network into smaller parts, allows the \graicore{} to manage and execute large models that do not fit on the \graicore{} at once.
This method processes one model part at a time, requiring only a part of the model to be loaded.
Larger AI models for inference typically offer higher accuracy and better generalization to new data, can capture more complex features, and are more robust to noise and distortions.
They are valuable for tasks requiring the utmost precision.

We aim to support seamless reconfiguration, which refers to the ability to switch between different AI models, or to modify the current model, without interrupting or significantly delaying data processing.
Given that the \graicore{} is focused on computer vision applications, this means that it is crucial that the processing of input images occurs in a timely manner.
This is particularly challenging for video applications, where each frame must be processed before the next one arrives.
Failing to meet this requirement can reduce the usefulness of the results, thereby degrading the system's quality of service.
Specifically, we aim the system to support an incoming video frame rate of up to \SI{60}{FPS}

To overcome this challenge, a high-speed interface can be introduced to connect to an off-chip memory.
While the \graicore{}'s on-chip SRAMs offer speed and energy efficiency, their limited size presents a bottleneck.
Off-chip memory like DRAM provides higher density but introduces higher latency and energy consumption.
Careful integration is crucial to maintain the energy efficiency, high speed, and low latency required for edge devices.

This thesis aims to adapt the architectural design of the \graicore{} to enable seamless reconfiguration.
This will be achieved through the following objectives:
\begin{itemize}
    \item
    Enhance the \graicore{}'s architecture for high speed data transfers from off-chip memory.
    This involves identifying necessary architectural modifications and proposing effective solutions to optimize data transfer rates.
    \item
    Enable the \graicore{} to support switching between multiple AI models for videos up to \SI{60}{FPS}.
    This requires an investigation of the data transfer speeds required to achieve timely model switching.
    \item
    Evaluate strategies for the \graicore{} to execute large AI models effectively.
    This includes exploring methods for partitioning large models into smaller, more manageable parts that can be efficiently processed by the \graicore{}.
    \item
    Ensure low-power operation of the \graicore{} during reconfiguration without compromising performance.
    This necessitates a comprehensive analysis of the system's energy consumption with the introduction of reconfigurations.
\end{itemize}