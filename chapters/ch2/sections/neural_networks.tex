\section{Deep learning}
Deep learning is a specialized branch of machine learning that focuses on training artificial neural networks with many layers.
These layers are the core building blocks that enable deep learning models to learn complex patterns and representations from vast amounts of data.
This approach has led to significant breakthroughs in various AI tasks, outperforming traditional machine learning methods in many applications \cite{}.

\begin{figure}[hbtp]
    \centering    
    \includegraphics[width=3cm]{example-image-a}
    \caption{TODO: classic ANN diagram}
\end{figure}

In the context of artificial neural networks, a layer is a collection of interconnected nodes, or neurons, that process information.
Each connection between two neurons has an associated weight (also known as a parameter), which is a numerical value that determines the strength of the connection.
These weights, play a crucial role in the learning process of a neural network.

\begin{figure}[hbtp]
    \centering    
    \includegraphics[width=3cm]{example-image-a}
    \caption{TODO: classic neuron diagram}
\end{figure}

Each layer performs a specific computation on the data it receives from the previous layer.
A neuron that receives input values from its connected neurons multiplies this value by their corresponding weights before being aggregated by the receiving neuron.
This aggregation is a simple summation, resulting in a weighted sum of the input values.
The neuron then applies a non-linear transformation to this weighted sum to produce its output.

A deep learning model typically consist of multiple layers stacked on top of each other, forming a hierarchical structure.
This structure allows the network to learn increasingly complex representations of the data as it flows through the layers.

Different types of layers serve different purposes in a deep learning model:
\begin{description}
    \item[Input layer:] 
    The first layer in the network, responsible for receiving the initial input data.
    Each neuron in this layer represent a feature of the input data, such as the pixel values in an image or the words in a sentence.
    \item[Hidden layer:] 
    Intermediate layers that lie between the input and output layers.
    They perform computations on the data received from the input, extracting features and learning increasingly abstract representations.
    Each hidden layer typically consists of many neurons, each applying a non-linear transformation (i.e., an activation function) to the weighted sum of its inputs.
    The activation function determines whether a neuron ``activates''.
    A neuron is said to be activated when its output is significant enough to contribute to the next layer's computations.
    Common activation functions used are ReLU, sigmoid and tanh.

    \begin{figure}[hbtp]
        \centering    
        \includegraphics[width=3cm]{example-image-a}
        \caption{TODO: activation functions}
    \end{figure}

    The number of hidden layers and neurons in each layer determines the depth and complexity of the network.
    \item[Output layer:] 
    The final layer, producing the result of the network's computation.
    The number of neurons in the output layer depends on the specific task, such as a classification label for image recognition or predicted value for natural language generation \cite{}.
\end{description}

ANNs learn by adjusting the weights of the connections between neurons.
This process is called training and involves feeding the network with labeled data and iteratively updating the weights to minimize the difference between the network's predictions and the true labels.
The most common algorithm for training ANNs is backpropagation \cite{}, which calculates the gradient of the error with respect to each weight and updates the weights accordingly.

\subsection{Convolutional neural network}
Convolutional neural networks (CNNs) are a specialized class of artificial neuron networks designed to process data with inherent spatial structure, such as images.

A key element that sets them apart is the convolutional layer.
CNNs leverage the concept of convolution to extract local features.
The network then uses these extracted features to construct increasingly complex understandings of the input as the data progresses through its layers

\begin{figure}[hbtp]
    \centering    
    \includegraphics[width=3cm]{example-image-a}
    \caption{TODO: convolution operation}
\end{figure}

A convolution operation involves sliding a learnable filter, or kernel, across the input data.
At each position, the filter performs an element-wise multiplication with the corresponding input data and sums the results.
This produces a single value in the output feature map.
By applying different filters, the network learns to detect various patterns like edges and shapes.

\begin{figure}[hbtp]
    \centering    
    \includegraphics[width=3cm]{example-image-a}
    \caption{TODO: pooling operation}
\end{figure}

The output of convolutional layers is then typically passed through pooling layers.
Pooling downsamples the feature map, reducing their dimensionality and making the network more robust to variations in the input.
Common pooling operations include taking the maximum or average value within a local region. % show pooling img

The use of activation functions introduces non-linearity, which enables the learning of complex patterns.
These activation functions are applied to the output of the convolutional and pooling layers.
% Common choices for such function are ReLU, sigmoid and tanh.

Finally, fully connected layers are often used in the final stages of a CNN.
These layers connect every neuron in the previous layer to every layer in the current layer.
This enables high-level reasoning and decision-making based on the extracted features.

The combining of simpler features into more complex ones across layers makes CNNs excel in computer vision tasks like image classification, object detection and image segmentation \cite{}.

\subsection{Residual neural network}

\begin{figure}[hbtp]
\centering    
\includegraphics[width=3cm]{example-image-a}
\caption{TODO: ResNet image}
\end{figure}

Artificial neural networks have shown success in various tasks, but training very deep networks can be challenging due to issues like vanishing gradients \cite{}.
Residual neural networks (ResNets) address this problem by introducing a novel architecture with skip connections, also known as residual connections.
These connections create ``shortcuts'' within the network, allowing information to bypass one or more layers.


\begin{figure}[hbtp]
\centering    
\includegraphics[width=3cm]{example-image-a}
\caption{TODO: ResNet image}
\end{figure}

\subsection{Model optimizations}
% TODO complete this section
% \begin{itemize}
%     \item Quantization
%     \item Pruning
%     \item ...
% \end{itemize}