Our previous analysis has highlighted a bottleneck in the \confignoc{}.
This bandwidth of the \confignoc{} hinders the overall performance and prevents us from executing multiple and large models in a timely manner.
To address this challenge, this chapter focuses on enhancing the \confignoc{}'s architecture to increase its bandwidth.

We explore a range of techniques to improve the \confignoc{}'s bandwidth:
\begin{description}
    \item[Improved packet format:]
    An improved packet format is explored to minimize the overhead during write transactions to the \graicore{}.
    \item[Widening link widths:]
    Widening of the link widths allows for more data to be transferred across a link in a single clock cycle.
    \item[Multiple injection points:]
    Adding additional injection points allows data to enter the \confignoc{} from multiple points in parallel.
\end{description}

This chapter provides an examination of each technique, analyzing their potential benefits and expected performance gains within the context of our \confignoc{} and application requirements.
