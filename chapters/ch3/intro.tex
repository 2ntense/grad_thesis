This chapter presents an analysis of the config NoC designed for the configuration of the GrAICore.
Another use of the config NoC is status monitoring reads while the GrAICore is in operation.
The config NoC is a separate NoC next to the event NoC, which is used for the execution of a model.
The config NoC plays a crucial role in facilitating data exchange between the various components within the system.

This analysis will investigate the config NoC's architecture, examining its key components and characteristics.
We will explore its topology, routing mechanisms, and communication protocols to understand how data is transferred and managed across the network.
Furthermore, the analysis will consider the efficiency of the applied format of the data packets.  
This investigation will allow us to assess the config NoC's effectiveness in meeting the system's requirements, particularly in terms of providing high bandwidth for achieving timely reconfigurations.

By thoroughly evaluating these aspects, we aim to provide a comprehensive understanding of the NoC's performance and suitability for its intended application.

% ? Do we analyze its power?