This chapter presents an analysis of the \confignoc{}, which is designed for the configuration of the \graicore{}.
Another use of the \confignoc{} is status monitoring reads while the \graicore{} is in operation.
The \confignoc{} is a separate NoC next to the \eventnoc{}, which is used for the execution of a model.
The \confignoc{} plays a crucial role in facilitating data exchange between the various components within the system.

This analysis investigates the \confignoc{}'s architecture, examining its key components and characteristics.
We will explore its topology, routing mechanisms, and communication protocols to understand how data is transferred and managed across the network.
Furthermore, the analysis will consider the efficiency of the applied format of the data packets that are transferred through the network.  
By thoroughly evaluating these aspects, this investigation allows us to assess the \confignoc{}'s effectiveness in meeting the system's requirements, particularly in terms of providing high bandwidth for achieving timely reconfigurations.
