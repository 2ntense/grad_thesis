\section{Bandwidth}
The \confignoc{}'s bandwidth is influenced by its unique characteristics.
While each link has a peak bandwidth of \SI{12.8}{Gb/s}\footnote{$\SI{16}{b} \times \SI{800}{MHz}$}, the effective bandwidth available for application data is closer to \SI{6.4}{Gb/s}.
This reduction is solely due to the 50\% overhead imposed by the 32-bit header in each packet.

Multi-hop transfers in a 2D mesh network typically suffer from increased latency as distance increases.
However, the use of wormhole switching, with its pipelined transmission of phits, minimizes this latency.
This ensures efficient data transfer regardless of the distance between nodes.

Furthermore, it is expected that in most cases the \confignoc{} is used for writing data to the \graicore{}.
The \confignoc{}'s use of a single injection point for write transactions offers an advantage.
This design inherently prevents multiple data sources from competing for the same network links, a common cause of contention and performance bottlenecks.

Configuration is done by transferring write request packets from the host to one or more neuron cores.
A write request packet consists of 64 bits in total.
Recall that the SRAM capacity of a single neuron core is \SI{256}{KiB}.
If we want to completely fill up one of these memories, a total of $\frac{\SI{256}{KiB}}{\SI{32}{b}} = \num{65536}$ write request packets need to be sent.
This means that, in total, \SI{256}{KiB} of payload data and \SI{256}{KiB} of header data is sent through the NoC.

To estimate the time to write a certain amount of data to the memories of the \graicore{}, we can make use of \cref{eq:latency}.
The effective bandwidth can be calculated with \cref{eq:bandwidth}.

\begin{equation} \label{eq:latency}
    T = 
    \frac{d_{\text{payload}} + d_{\text{overhead}}}
    {f_{\text{clock}} \times w_{\text{phit}}}
\end{equation}

\begin{equation} \label{eq:bandwidth}
    \text{BW}_{\text{eff}} =
    \frac{d_\text{payload}}{T}
\end{equation}

\begin{eqexpl}[15mm]
    \item{$T$} time to write data
    \item{$\text{BW}_{\text{eff}}$} effective bandwidth
    \item{$d_{\text{payload}}$} amount of payload data
    \item{$d_{\text{overhead}}$} amount of overhead data
    \item{$f_{\text{clock}}$} system's clock frequency
    \item{$w_{\text{phit}}$} width of a phit
\end{eqexpl}

For example, to completely fill up the \graicore{}'s memory (\SI{36}{MiB}), it will take around \SI{47}{ms}\footnote{$\frac{\SI{36}{MiB} + \SI{36}{MiB}}{\SI{800}{MHz} \times \SI{16}{b}}$}.
This translates to an effective bandwidth of \SI{6.4}{Gb/s} or \SI{763}{MiB/s}\footnote{$\frac{\SI{36}{MiB}}{\SI{47}{ms}}$}, similar to the effective bandwidth for a single link.
Here we assumed that no delays are introduced by software and reading the data from external storage.
Furthermore, delays introduced by hops through the \confignoc{} are minimal due to pipelining.
Thus, \cref{eq:bandwidth} provides us the \confignoc{}'s peak effective bandwidth.
