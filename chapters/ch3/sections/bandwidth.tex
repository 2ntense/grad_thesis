\section{Bandwidth}
Configuration is done by transferring write request packets from the host to one or more neuron cores.
A write request packet consists of 64 bits in total.
Recall that the SRAM capacity of a single neuron core is \SI{256}{KiB}.
If we want to completely fill up one of these memories, a total of $\frac{\SI{256}{KiB}}{\SI{32}{b}} = 65536$ write request packets needs to be sent.
This means that, in total, \SI{256}{KiB} of payload data and \SI{256}{KiB} of header data is sent through the NoC.

To estimate the time to write a certain amount of data to the memories of the GrAICore, we can make use of \cref{eq:latency}.
The effective bandwidth can be calculated with \cref{eq:bandwidth}.

\begin{equation}
    T = 
    \frac{d_{\text{payload}} + d_{\text{overhead}}}
    {f_{\text{clock}} \times w_{\text{phit}}}
\label{eq:latency}
\end{equation}

\begin{equation}
    \text{BW}_{\text{eff}} =
    \frac{d_\text{payload}}{T}
\label{eq:bandwidth}
\end{equation}

\begin{eqexpl}[15mm]
    \item{$T$} time to write data
    \item{$\text{BW}_{\text{eff}}$} effective bandwidth
    \item{$d_{\text{payload}}$} amount of payload data
    \item{$d_{\text{overhead}}$} amount of overhead data
    \item{$f_{\text{clock}}$} system's clock frequency
    \item{$w_{\text{phit}}$} width of a phit
\end{eqexpl}

For example, to completely fill up the GrAICore's memory (\SI{36}{MiB}), it will take around \SI{47}{ms}\footnote{$\frac{\SI{36}{MiB} + \SI{36}{MiB}}{\SI{800}{MHz} \times \SI{16}{b}}$}.
% This translates to an effective bandwidth of $\frac{\SI{36}{MiB}}{\SI{47}{ms}} \approx \SI{763}{MiB/s}$.
This translates to an effective bandwidth of \SI{763}{MiB/s}\footnote{$\frac{\SI{36}{MiB}}{\SI{47}{ms}}$}.
Here we assumed that no delays are introduced by software and reading the data from storage.
Furthermore, it is good to note that the delays introduced by hops are minimal due to pipelining.
Thus, \cref{eq:bandwidth} provides us the config NoC's \textit{peak} effective bandwidth.
