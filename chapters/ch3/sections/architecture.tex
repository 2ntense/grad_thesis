\section{Architecture}
The config NoC incorporates a 2D mesh network for communication across all the routers and neuron cores.
An extra link is attached to the config NoC that enables interfacing with the external world.
Compared to the event NoC, intercommunication does not occur between neuron cores.

% TODO To make this description even clearer, you could create a simple diagram showing the 2D mesh of routers with arrows indicating the XY routing directions

The communication links between nodes are 16-bits wide, determining the amount of data that can be transferred concurrently.
The config NoC uses a packet-switched communication protocol, where data is transmitted in the form of independent packets routed through the network.

For routing, XY routing, a deterministic algorithm is employed.
Packets first travel horizontally along the X-dimension until they reach the destination column, and then vertically along the Y-dimension to the destination row.
At each router, wormhole switching is utilized.
This is a low-latency technique where packets are divided into smaller units called flits.
The header flit establishes the path, and the remaining flits follow in a pipelined manner, minimizing delays as the packet ''burrows`` through the network.
A key characteristic of the config NoC's architecture is its \SI{800}{MHz} clock frequency.
This clock governs the rate of data transmission across the network, influencing the overall performance and responsiveness of the on-chip communication.

Unlike the config NoC, the event NoC uses wraparound links that helps to drastically reduce the average amount of hops a packet has to take.
Wraparound links do not provide any noticeable benefits to the config NoC since these interactions do not occur with the config NoC.
Most of the high traffic data transaction occurs when data is written to the neuron cores' SRAMs.
This process only injects the data in the config NoC via a single point.

% TODO add clock frequency
% * Link bandwidth