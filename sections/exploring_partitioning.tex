% \section{Introduction}
Suppose we have a large model that does not fit on the GrAICore at once.
We will look at two strategies to partition the model to allow it to be executed on the GrAICore.

Assume we have a model of that requires \SI{60}{MiB} to be written to the GrAICore.
We are able to partition this model in two parts of \SI{30}{MiB} and four parts of \SI{15}{MiB}.
Furthermore, assume the latencies for processing and configuring as shown in \cref{tab:partitiong_strategies_example}.

\begin{table}[hbtp]
\centering
\begin{tabular}{@{}lll@{}}
\toprule
\textbf{Part} & \textbf{Processing (ms)} & \textbf{Configuration (ms)} \\ \midrule
\SI{30}{MiB}  & 4                        & 8                           \\
\SI{15}{MiB}  & 2                        & 4                          
\end{tabular}
\caption{TODO}
\label{tab:partitiong_strategies_example}
\end{table}

\section{Sequential configuration}
With the model partitioned in two parts of \SI{30}{MiB}, we can only fit one part on the GrAICore at a time.
Because of this, we can only configure and process in a sequential manner.
The configure and process timeline is illustrated in \cref{fig:schedule_sequential_configuration}.
% In total, to fully process the model in two parts, it takes \SI{16}{ms}.
The inference latency of a single input frame is \SI{16}{ms}.
\Cref{fig:schedule_sequential_configuration_extended} shows a timeline when processing multiple input frames in sequence, we see that we can reach a stable input frame rate of \SI{41.7}{FPS} ($\frac{1}{\SI{24}{ms}}$).

\begin{figure}[hbtp]
    \centering
    \resizebox{0.6\linewidth}{!}{
        \begin{RTGrid}[
    width=10cm,
    nosymbols=1,
]{2}{26}
    \RowLabel{1}{Configure}
    \RowLabel{2}{Process}

    \TaskExecution[execlabel=$p_1$,color=lightgray]{1}{0}{8}
    \TaskExecution[execlabel=$p_2$,color=lightgray]{1}{12}{20}

    \TaskExecution[execlabel=$p_1$,color=lightgray]{2}{8}{12}
    \TaskExecution[execlabel=$p_2$,color=lightgray]{2}{20}{24}
\end{RTGrid}
    }
    \caption{Example timeline of sequential configuration and processing}
    \label{fig:schedule_sequential_configuration}
\end{figure}

\begin{figure}[hbtp]
    \centering
    \resizebox{1.0\linewidth}{!}{
        \begin{RTGrid}[
    width=20cm,
    nosymbols=1,
]{2}{52}
    \RowLabel{1}{Configure}
    \RowLabel{2}{Process}

    \TaskExecution[execlabel=$p_1$,color=lightred]{1}{0}{8}
    \TaskExecution[execlabel=$p_2$,color=lightred]{1}{12}{20}

    \TaskExecution[execlabel=$p_1$,color=lightred]{2}{8}{12}
    \TaskExecution[execlabel=$p_2$,color=lightred]{2}{20}{24}

    \TaskExecution[execlabel=$p_1$,color=lightcyan]{1}{24}{32}
    \TaskExecution[execlabel=$p_2$,color=lightcyan]{1}{36}{44}

    \TaskExecution[execlabel=$p_1$,color=lightcyan]{2}{32}{36}
    \TaskExecution[execlabel=$p_2$,color=lightcyan]{2}{44}{48}

    \TaskRespTime{2}{0}{8}
    \TaskRespTime{2}{24}{8}
    
    % \TaskArrival{2}{8}
    \TaskDeadline{2}{24}
    % \TaskArrival{2}{32}
    \TaskDeadline{2}{48}
\end{RTGrid}
    }
    \caption{
    Example timeline of sequential configuration and processing for two input frames.
    The shaded areas represent periods where we expect an input frame.
    The arrows pointing down represent the points of time where the associated input frame has been fully processed.
    }
    \label{fig:schedule_sequential_configuration_extended}
\end{figure}

\section{Parallel configuration}
With the model partitioned in four parts, two parts of \SI{15}{MiB} can fit on the GrAICore at the same time.
This allows for a pipeline of configuration and processing.
For this approach to work, the neuron cores must be divided into two isolated segments/zones.
% With each zone able to contain a part.
Each of the zones will contain a single part.
With the introduction of the additional injection point, two model parts can be written to the local memories of the neuron cores in parallel.

\begin{figure}[hbtp]
    \centering
    \resizebox{0.65\linewidth}{!}{
        \begin{RTGrid}[
    width=10cm,
    nosymbols=1,
]{4}{26}
    \RowLabel{1}{Configure $Z_1$}
    \RowLabel{2}{Configure $Z_2$}
    \RowLabel{3}{Process $Z_1$}
    \RowLabel{4}{Process $Z_2$}

    \TaskExecution[execlabel=$p_1$,color=lightgray]{1}{0}{4}
    \TaskExecution[execlabel=$p_3$,color=lightgray]{1}{6}{10}

    \TaskExecution[execlabel=$p_2$,color=lightgray]{2}{0}{4}
    \TaskExecution[execlabel=$p_4$,color=lightgray]{2}{8}{12}

    \TaskExecution[execlabel=$p_1$,color=lightgray]{3}{4}{6}
    \TaskExecution[execlabel=$p_3$,color=lightgray]{3}{10}{12}

    \TaskExecution[execlabel=$p_2$,color=lightgray]{4}{6}{8}
    \TaskExecution[execlabel=$p_4$,color=lightgray]{4}{12}{14}
\end{RTGrid}
    }
    \caption{Example timeline of parallel configuration and processing}
    \label{fig:schedule_parallel_configuration}
\end{figure}

% This approach allows configuration of the two ``zones'' to be performed in parallel (see \cref{fig:schedule_parallel_configuration}).
% Also, processing and configuration can happen in parallel.
\Cref{fig:schedule_parallel_configuration} shows a timeline of a pipelined configuration and processing.
With this approach, the a constraint is that while the system is processing part $p_n$, the system should not reconfigure the zone where part $p_n$ resides at the same time.
$Z_1$ and $Z_2$ indicate one of the two independent zones.
Note that processing cannot be performed in parallel due to the input of the next part being dependent on the previous part's output.
% The following figure shows the timeline using ASAP scheduling for the tasks.
Observe that the configuration of $p_2$ can start \SI{2}{ms} later without affecting the total inference latency.
The total inference latency of this approach is \SI{10}{ms}.
This is a speedup of $1.6\times$ compared to the sequential configuration approach.
% with \SI{100}{FPS} ($\frac{1}{\SI{10}{ms}}$) as the highest manageable input frame rate.
% However, this does not allow for an input frame rate of \SI{100}{FPS} ($\frac{1}{\SI{10}{ms}}$).

\Cref{fig:schedule_parallel_configuration_extended} shows a possible timeline for three consecutive input frames at the highest manageable frame rate.
An arrow pointing up shows the arrival of an input frame.
It is apparent that the highest manageable frame rate is \SI{83.3}{FPS} ($\frac{1}{\SI{12}{ms}}$).

\begin{figure}[hbtp]
    \centering
    \resizebox{0.8\linewidth}{!}{
        \begin{RTGrid}[
    width=15cm,
    nosymbols=1,
]{4}{38}
    \RowLabel{1}{Configure $Z_1$}
    \RowLabel{2}{Configure $Z_2$}
    \RowLabel{3}{Process $Z_1$}
    \RowLabel{4}{Process $Z_2$}

    \TaskExecution[execlabel=$p_1$, color=lightred]{1}{0}{4}
    \TaskExecution[execlabel=$p_3$, color=lightred]{1}{6}{10}

    \TaskExecution[execlabel=$p_2$, color=lightred]{2}{0}{4}
    \TaskExecution[execlabel=$p_4$, color=lightred]{2}{8}{12}

    \TaskExecution[execlabel=$p_1$, color=lightred]{3}{4}{6}
    \TaskExecution[execlabel=$p_3$, color=lightred]{3}{10}{12}

    \TaskExecution[execlabel=$p_2$, color=lightred]{4}{6}{8}
    \TaskExecution[execlabel=$p_4$, color=lightred]{4}{12}{14}
    %%%
    \TaskExecution[execlabel=$p_1$, color=lightcyan]{1}{12}{16}
    \TaskExecution[execlabel=$p_2$, color=lightcyan]{2}{14}{18}

    \TaskExecution[execlabel=$p_1$, color=lightcyan]{3}{16}{18}
    \TaskExecution[execlabel=$p_2$, color=lightcyan]{4}{18}{20}

    \TaskExecution[execlabel=$p_3$, color=lightcyan]{1}{18}{22}
    \TaskExecution[execlabel=$p_4$, color=lightcyan]{2}{20}{24}

    \TaskExecution[execlabel=$p_3$, color=lightcyan]{3}{22}{24}
    \TaskExecution[execlabel=$p_4$, color=lightcyan]{4}{24}{26}
    %%%
    \TaskExecution[execlabel=$p_1$, color=lightgreen]{1}{24}{28}
    \TaskExecution[execlabel=$p_2$, color=lightgreen]{2}{26}{30}

    \TaskExecution[execlabel=$p_1$, color=lightgreen]{3}{28}{30}
    \TaskExecution[execlabel=$p_2$, color=lightgreen]{4}{30}{32}

    \TaskExecution[execlabel=$p_3$, color=lightgreen]{1}{30}{34}
    \TaskExecution[execlabel=$p_4$, color=lightgreen]{2}{32}{36}

    \TaskExecution[execlabel=$p_3$, color=lightgreen]{3}{34}{36}
    \TaskExecution[execlabel=$p_4$, color=lightgreen]{4}{36}{38}
    %%%
    % \TaskArrival{3}{4}
    % \TaskArrival{3}{16}
    % \TaskArrival{3}{28}
    
    \TaskRespTime{3}{0}{4}
    \TaskRespTime{3}{14}{2}
    \TaskRespTime{3}{26}{2}
    
    \TaskDeadline{4}{14}
    \TaskDeadline{4}{26}
    \TaskDeadline{4}{38}
\end{RTGrid}

    }
    \caption{
    Example timeline of parallel configuration and processing for three input frames.
    The shaded areas represent periods where we expect an input frame.
    The arrows pointing down represent the points of time where the associated input frame has been fully processed.
    }
    \label{fig:schedule_parallel_configuration_extended}
\end{figure}

