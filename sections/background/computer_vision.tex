\section{Computer vision}

Computer vision (CV) is a field of artificial intelligence (AI) that enables machines the ability to interpret, analyze and understand visual data from images and videos.
CV seeks to replicate the visual capabilities of the human vision, allowing bridging of the gap between human perception and AI \cref{TODO}.
It has become an extremely important part of AI, finding applications in diverse fields such as healthcare, security, retail, agriculture, entertainment and robotics.
CV systems rely on various algorithms such as image processing, machine learning and deep learning for the analysis and processing of visual data.
Recent advancements in deep learning such as convolutional neural networks (CNNs) have revolutionized the field, allowing machines to achieve human-like performance in visual processing or even exceeding it \cref{TODO}.

In order to process images, they are first transformed to a digital image, which is a collection of numerical values representing the color at a particular point of the image plane.
These points are called pixels.
In most cases, each pixel possesses three numerical values representing the red, green and blue color intensities.

Analysis of an image by CV systems generally consists of the following steps:
\begin{enumerate}
    \item Image acquisition
    \item Image preprocessing
    \item Image representation
    \item Machine learning and deep learning
    \item Object detection and recognition
\end{enumerate}



