\begin{tikzpicture}[
    input/.style={matrix of nodes,
          nodes={draw, minimum size=5mm, fill=green!30},
          column sep=-\pgflinewidth, row sep=-\pgflinewidth,
          nodes in empty cells,
          },
    pool/.style={matrix of nodes,
          nodes={draw, minimum size=5mm, fill=blue!30},
          column sep=-\pgflinewidth, row sep=-\pgflinewidth,
          nodes in empty cells,
          },
  ]
  
  % Input matrix (feature map)
  \matrix[input] (mat1)
  {
  1 & 2 & 2\\
  0 & 3 & 4\\
  5 & 1 & 2\\
  };
  \node[below=0.1cm of mat1] {Feature Map};
  
  % Pooling matrix (moved higher)
  \matrix[pool, above right=0cm and 1.5cm of mat1] (mat2)
  {
  3 & 4  \\
  5 & 4 \\
  };
  \node[below=0.2cm of mat2] {Pooled Feature Map};
  
  % Draw lines
  \draw[red!30, thin] (mat1-1-1.north west) -- (mat2-1-1.north west);
  \draw[red!30, thin] (mat1-2-2.south east) -- (mat2-1-1.south east);
  
  
  % Highlight the pooling window
  \draw[red, thick, dashed] (mat1-1-1.north west) -- (mat1-1-2.north east) -- (mat1-2-2.south east) -- (mat1-2-1.south west) -- cycle;
  
\end{tikzpicture}
