\begin{tikzpicture}[
    input/.style={matrix of nodes,
          nodes={draw, minimum size=5mm, fill=yellow!30},
          column sep=-\pgflinewidth, row sep=-\pgflinewidth,
          nodes in empty cells,
          },
    kernel/.style={matrix of nodes,
          nodes={draw, minimum size=5mm, fill=red!30},
          column sep=-\pgflinewidth, row sep=-\pgflinewidth,
          nodes in empty cells,
          },
    feature/.style={matrix of nodes,
          nodes={draw, minimum size=5mm, fill=green!30},
          column sep=-\pgflinewidth, row sep=-\pgflinewidth,
          nodes in empty cells,
          },
  ]
  
  % Input matrix
  \matrix[input] (mat1)
  {
  1 & 2 & 1 & 0\\
  0 & 1 & 1 & 1\\
  3 & 0 & 1 & 2\\
  1 & 1 & 0 & 1\\
  };
  \node[below=0.1cm of mat1] {Input};
  
  % Kernel matrix
  \matrix[kernel, above right=0.5cm and 1cm of mat1] (mat2)
  {
  0 & 1 & 0\\
  1 & 1 & 1\\
  0 & 1 & 0\\
  };
  \node[above=0.1cm of mat2] {Kernel};
  
  % Feature map matrix
  \matrix[feature, below right=0.5cm and 1cm of mat2] (mat3)
  {
  4 &  \\
   &  \\
  };
  \node[below=0.1cm of mat3] {Feature Map};
  
  % Draw lines
  \draw[blue!30, thin] (mat1-1-1.north west) -- (mat2-1-1.north west);
  \draw[blue!30, thin] (mat1-3-3.south east) -- (mat2-3-3.south east);
  
  % Highlight the receptive field
  \draw[red, thick, dashed] (mat1-1-1.north west) -- (mat1-1-3.north east) -- (mat1-3-3.south east) -- (mat1-3-1.south west) -- cycle;
  
\end{tikzpicture}
